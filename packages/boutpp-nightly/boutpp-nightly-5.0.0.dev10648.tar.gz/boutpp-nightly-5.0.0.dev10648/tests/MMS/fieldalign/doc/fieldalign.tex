\documentclass[12pt,a4paper]{article}
\usepackage{calc}% http://ctan.org/pkg/calc
\usepackage[
  letterpaper,%
  textheight={47\baselineskip+\topskip},%
  textwidth={\paperwidth-126pt},%
  footskip=75pt,%
  marginparwidth=0pt,%
  top={\topskip+0.75in}]{geometry}% http://ctan.org/pkg/geometry
\usepackage{fancyhdr}% http://ctan.org/pkg/fancyhdr

\newcommand{\kpar}{k_{||}}
\newcommand{\kperp}{k_\perp}

\pagestyle{fancy}
%\fancyhead{}
\fancyfoot{}% Clear header & footer
%\fancyhead[L]{Electrostatic Ohm's law}% Set centered header
%\fancyhead[R]{ 22$^{nd}$ March 2014 }
\fancyfoot[C]{\thepage}% Set centered footer
\renewcommand{\headrulewidth}{0.5pt}% Add header rule

% Titling
\title{ Testing non-orthogonal coordinates }% Your title
\author{ Ben Dudson and Jarrod Leddy }% Your author
\date{ 24$^{th}$ March 2015 }% Your date
\makeatletter
% Definition of proc.cls' \maketitle:
\def\@maketitle{% 
  \vbox to 1.25in{%
    \vskip 0.5em
    {\large \begin{tabular}[t]{ll}
        Title: & \@title \\
        Author: & \@author \\
        Date: & \@date \\
        Project: & 
      \end{tabular}\par}
    \vfil}\noindent\hrulefill
    \vskip 2em}
\makeatother
\begin{document}
\maketitle % Produce title
\thispagestyle{fancy}% Override titling page style with fancy

Solving the equation
\[
\frac{\partial f}{\partial t} = \frac{\partial f}{\partial \psi} + \frac{\partial f}{\partial \theta} + \frac{\partial f}{\partial \phi} = \left(\hat{\bf{e}}_\psi + \hat{\bf{e}}_\theta + \hat{\bf{e}}_\phi\right)\cdot\nabla f
\]
on a 3D $\left(\psi, \theta, \phi\right)$.\\

Transform into coordinates
\begin{eqnarray*}
x &=& \psi \\
y &=& \theta - \int_{\psi_0}^\psi \eta d\psi  \\
z &=& \phi - \int_{y_0}^y \nu\left( 1 + \int_{\psi_0}^{\psi} \eta d\psi \right) dy
\end{eqnarray*}
with pitch $\nu$ and poloidal nonorthogonality parameter $\eta$. Taking the gradient of these expressions, and
using $\nabla = \nabla x \frac{\partial}{\partial x} + \nabla y \frac{\partial}{\partial y} + \nabla z\frac{\partial}{\partial z}$ assuming
\[
\left|\nabla\psi\right| = 1 \qquad \left|\nabla\theta\right| = 1 \qquad \left|\nabla\phi\right| = 1
\]
gives
\begin{eqnarray*}
\nabla x &=& \nabla \psi \\
\nabla y &=& G\nabla \theta - \eta \nabla \psi \\
\nabla z &=& \nabla \phi - H \nabla \theta - I\nabla\psi
\end{eqnarray*}
with
\begin{eqnarray*}
G &=& 1 - \frac{\partial}{\partial\theta}\int_{x_0}^x \eta dx \\
I &=& \frac{\partial}{\partial\psi}\int_{y_0}^y \nu\left(1+ \frac{\partial}{\partial y}\int_{x_0}^x \eta dx \right) dy \\
I &=& \frac{\partial}{\partial\theta}\int_{y_0}^y \nu\left(1+ \frac{\partial}{\partial y}\int_{x_0}^x \eta dx \right) dy
\end{eqnarray*}
The evolution equation becomes
\[
\frac{\partial f}{\partial t} = \frac{1}{G}\left[ \left( G + \eta+GI+H\eta \right)\nabla x + \left(1 + H\right) \nabla y + G\nabla z \right] \cdot \nabla f
\]
which when expanded gives
\begin{eqnarray*}
\frac{\partial f}{\partial t} &=& \underbrace{\frac{G + \eta+GI+H\eta}{G}}_{v_x}\left[g^{xx} \frac{\partial f}{\partial x} + g^{xy} \frac{\partial f}{\partial y} + g^{xz}\frac{\partial f}{\partial z}\right] \\
&+& \underbrace{\frac{H + \nu}{G}}_{v_y}\left[g^{xy} \frac{\partial f}{\partial x} + g^{yy} \frac{\partial f}{\partial y} + g^{yz}\frac{\partial f}{\partial z}\right] \\
&+& \underbrace{1}_{v_z}\left[g^{xz} \frac{\partial f}{\partial x} + g^{yz}\frac{\partial f}{\partial y} + g^{zz} \frac{\partial f}{\partial z}\right]
\end{eqnarray*}
with
\[
g^{xx} = 1 \qquad g^{yy} = G^2+\eta^2 \qquad g^{zz} = 1 + H^2 + I^2 \qquad g^{xy} = -\eta \qquad g^{yz} = \eta I - GH \qquad g^{xz} = -I 
\]

\end{document}
